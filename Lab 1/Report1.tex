\documentclass{perassignments}



\usepackage{hyperref}
\usepackage[abjad]{pertheorems}
\usepackage{mathtools}
\usepackage{amsmath}
\usepackage{float}
\usepackage{common}
\usepackage{xepersian}




\settextfont[]{XBNiloofar}
%\setmathdigitfont{XBTabriz}


\CourseName[آز شبکه‌]
\Semester[اول]
\Year[01]
\Prof[دکتر صفائی]
\Dept[دانشکده مهندسی کامپیوتر]
\CollabFirst[عماد زین‌اوقلی]{98103267}
\CollabSecond[پارسا رییسی]{98103223}
\CollabThird[سیدابوالفضل رحیمی ]{97105941}

\renewcommand{\maketitle}{\MakeMyLabTitle}
\allowdisplaybreaks

\begin{document}
	\maketitle
	\begin{enumerate}
		\item \
		
		\begin{table}[H]
			\centering 
			\begin{tabular}{c|c|c|c}
				& زوج سیم بهم‌تافته & کابل هم‌محور & فیبر نوری\\ \hline
				سرعت انتقال‌داده & 
				نسبتا کم:
				\lr{upto 1 Gbps} &
				متوسط:
				 \lr{upto 10 Gbps} & 
				 بالا:
				 \lr{upto 100 Gbps}\\ \hline 
				احتمال ایجاد خطا & بالا & کم & بسیار کم\\ \hline 
				میزان کاهش انرژی سیگنال & 
				\lr{200 dB/km} & \lr{100 dB/km}  & \lr{0.1 dB/km} \\ \hline 
			\end{tabular}
		\end{table}
		با توجه به جدول بالا می‌توان گفت که زوج سیم بهم‌تافته برای کابرد‌های کم هزینه که نیاز‌مند پنهای زیاد و سرعت بالا و سیم‌های طویل نیستند، مناسب‌تر هستند. کابل‌ هم‌محور با توجه به هزینه کم‌تر و آسانی پیاده‌سازی نسبت به فیبر نوری معمولا در کابرد‌های معمولی همجون کاربرد‌های خانگی مثل تلویزیون و دوربین‌های مدار بسته استفاده می‌شوند.
	در نهایت به دلیل هزینه بالا و سختی پیاده‌سازی، فیبر‌های نوری معمولا برای ارتباطات بلند همچون خط‌های بین قاره‌ای و یا در دیتاسنتر‌ها استفاده می‌شود.
		\item 
		مدل TCP/IP دارای چهار لایه است.
		\begin{description}
			\item[لایه کابرد]
			در این لایه پروتکل‌ها و رابط‌های مورد نیاز برنامه‌ها و کاربرها فراهم می‌شوند. به عنوان مثال پروتکل‌های FTP و HTTP که نحوه انتقال داده و پیام در شبکه را تعیین می‌کند در این لایه قرار دارد. 
			\item[لایه انتقال]
			سرویس ارتباط و انتقال‌داده میزبان-میزبان است را فراهم می‌کند. دو پروتکل TCP و UDP در این لایه قرار دارند که با ایجاد ارتباط بین دو میزبان به انتقال داده و پیام بین این دو میزبان می‌پردازند.
			\item[لایه اینترنت]
			سرویس‌های مربوط به انتقال منطقی داده همچون آدرس‌دهی و پیدا کردن مسیر بهینه را فراهم می‌کند. پروتکل IP در این لایه قرار دارد.
			\item[لایه لینک]
			سرویس مربوط به انتقال داده بر روی محیط فیزیکی شبکه مثل انتقال داده بر روی لینک‌های میزبان-روتر و روتر-روتر‌ را فراهم می‌کند.
		 \end{description}
	 	لایه کابرد TCP/IP معادل با لایه کابرد، نمایش و نشست OSI است. لایه انتقال و شبکه در هر دو مدل یکی است. در نهایت لایه لینک در مدل TCP/IP معادل با لایه انتقال‌داده و لایه فیزیکی مدل OSI است. به طوری کلی‌تر، مدل OSI، تنها بر اساس وظیفه‌های هر لایه تقسیم‌بندی شده و مستقل از پروتکل است. مدل TCP/IP یک پیاده‌سازی از مدل OSI که وابسته به پروتکل است.
		\item 
		کابل‌های کراس برای اتصال دو دستگاه یکسان همچون اتصال کامپیوتر به کامپیوتر استفاده می‌شوند. در عوض کابل‌های مستقیم برای اتصال دو دستگاه مختلف همچون اتصال کامپیوتر به سوییچ استفاده می‌شوند. در قدیم، استفاده از کابل کراس برای اتصال دو کامپیوتر ضروری بوده است. اما امروز اغلب دستگاه‌های شبکه قادر به تشخیص نوع کابل و پین‌های درست هستند.
	\end{enumerate}
\end{document}